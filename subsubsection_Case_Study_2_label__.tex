\subsubsection{Case Study #2}
\label{casestudy2}

The second case study is a campus in the Northeast region of the United States. It is also a University and it has 180 buildings on a single main campus. An initial  meeting was organized in April 2015 with the facilities management team. This campus has well-organized building and energy management systems with a strong emphasis on data acquisition and management. The campus has an analytics and automated fault detection software platform that is connected to the underlying controls systems. A follow-up campus visit was conducted in August 2015 to facilitate the download of a raw, example data set from the buildings on campus. At this point, a log-in to a new data management platform was given for the purposes data extraction. Several issues arose from the use of this platform and ultimately, a database query by the software developers of the system was used to extract the one year of electrical meter data from the campus buildings. Once again, a spreadsheet of meta-data was shared that included information on floor area and primary building use type. A final site visit was conducted in April 2016 to discuss some of the results of the data acquisition and upcoming plans for upgrades. A formal presentation of the results was not able to be given, thus only limited feedback of the implementation progress was collected.