\subsection{Anomaly Detection}
\label{AnomalyDetection}
Anomaly detection for buildings focuses on the detection and diagnostics of problems occurring within a building, its subsystems, and components. This field is most often focuses on the use of novelty detection or clustering approaches to find anomalous behavior. The sub-categories for this section are divided according to the spatial hierarchy of systems within a building; the highest level is whole building consumption, down to the subsystems such as heating, cooling or lighting and then to the individual components of those systems. 

\subsubsection{Whole Building}
Whole building anomaly detection uses the electricity or heating and cooling energy supply in coming to a building to determine sub-sequences of poor performance. This category is complimentary to many of the Smart Meter solutions as they both focus on the use of a single data stream for a building. Seem \cite{seem_using_2006} had an early work again in this category with his work in using novelty detection to find abnormal days of consumption in buildings. Liu et al. \cite{liu_method_2010} used classification and regression trees (CART) and Wrinch et al. \cite{wrinch_anomaly_2012} frequency domain analysis for the same purpose. Jacob et al. utilized hierarchical clustering to use as variables in regression models for whole building monitoring \cite{jacob_black-box_2010}. Fontugne et al. \cite{fontugne_strip_2013} created a process known as the \emph{Strip, Bind, and Search} method to automatically uncover misbehavior from the whole building level and subsequently detects the source of the anomaly. Janetzko et al. developed a visual analytics platform to highlight anomalous behavior in power meter data \cite{janetzko_anomaly_2013}. Chou and Telaga created a hybrid whole building anomaly detection process using K-means \cite{chou_real-time_2014}. Ploennigs et al. \cite{ploennigs_exploiting_2013} and Chen et al. \cite{chen_statistical_2014} created similar systems that use generalized additive models (GAM). In the most recent work, Capozzoli et al. \cite{capozzoli_fault_2015} and Fan et al. \cite{fan_framework_2015} use various techniques as part of a framework to detect and diagnose performance problems. 

\subsubsection{Subsystems}
Subsystem anomaly detection focuses on the use of a broader data set to detect and diagnose faults from a lower level. Yoshida et al. \cite{yoshida_identification_2008} provided a semi-supervised approach that seeks to determine which variables within a building are most influential in contributing to overall building performance. Wang et al. \cite{wang_system-level_2010} uses PCA to diagnose sensor failures. Forlines and Wittenberg visualized multi-dimensional data using what they call the Wakame diagram \cite{forlines_wakame:_2010}. Linda et al. \cite{linda_computational_2012} and Wijayasekara et al. \cite{wijayasekara_mining_2014} use various techniques to diagnose system faults and visualize them spatially. Le Cam et al. \cite{le_cam_application_2014} use PCA to create inverse models to detect problems in HVAC systems. Li and Wen \cite{li_model-based_2014} created a similar process using PCA in conjunction with wavelet transform.  Sun et al. \cite{sun_efficient_2015} used data association rules to create fault detection thresholds for finding anomalies.

\subsubsection{Components}
Component level anomaly detection is a bottom-up fault detection approach that focuses on determining faults in individual equipment. Wang and Cui \cite{wang_sensor-fault_2005} use PCA to detect component faults in chilled water plants. Yu et al. \cite{yu_novel_2012} and Fontugne et al. \cite{fontugne_mining_2013} both compliment their work at the whole building level to find associated component performance anomalies automatically. Zhu et al. \cite{zhu_fault_2012} use wavelets to diagnose issues in air handling units (AHU).
