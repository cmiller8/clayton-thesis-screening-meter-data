\section{Conclusion and Outlook}
\label{sec:conclusion}

The most informative features are X. (include a table that outlines the most informative features)

The conclusions from this study are:

\begin{itemize}
\item Characterization of various use types 
\end{itemize}

% \subsection{Research Questions}
% The primary question addressed through this research is:
% \begin{itemize}
% \item How much information about a building can be predicted solely through the analysis of raw hourly or sub-hourly, whole building electrical meter data with a scarcity of conventional characteristic data? 
% \end{itemize}
% This question is dissected into several more specific parts:
% \begin{itemize}
% \item Which temporal features are most accurate in classifying the primary use-types of a building?
% \item Can temporal features be used to better benchmark buildings by signifying how \emph{well a building fits within its designated use-type class}?
% \item Can temporal features be used to forecast whether an energy savings intervention measure will be successful or not?
% \item What are the most appropriate parameter settings for various generalized temporal feature extraction techniques as applied to this context?
% \item Is it effective or possible to implement such features across data from tens of thousands of buildings?
% \end{itemize}
