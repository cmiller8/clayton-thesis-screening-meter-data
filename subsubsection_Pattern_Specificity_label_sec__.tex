\subsubsection{Pattern Specificity}
\label{sec:patternspec}

Another way to leverage SAX to characterize the case study data is to use it to extract which patterns are most indicative of a certain building use type. This information is extracted using the SAX-VSM process pioneered by Senin and Malinchik that uses SAX and Vector Space Model technique from the text mining field \cite{Senin_2013}. Conventionally this technique is utilized as a classification model to predict which class a certain time-series belongs to. A by-product of the process is that the subsequences of each data stream are assigned a metric indicating their specificity. Pattern specificity is a concept that quantifies how well a meter \emph{fits within its own class}. This technique is used to determine whether a building is actually operating similar to other supposed peer buildings of the same type.

The SAX-VSM process begins with the SAX word creation, similar to \emph{DayFilter} as shown in Figure \ref{fig:saxcreation}. However, the key difference is that the conventional SAX process extracts word patterns from overlapping windows as opposed to simply \emph{chunking} each daily profile. Each data stream within a certain class of a the training data set is converted to SAX words using the same input variables of alphabet size, $A$, and subsequence period count, $W$. In addition, a $P$ variable is chosen to indicate the size of the sliding window. 