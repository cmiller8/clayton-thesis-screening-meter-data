\subsubsection{University Dormitory and Laboratory Comparison}
\label{sec:dormvslab}

The random forest classification model and variable importance metrics provide an indication of how the features characterize the a building's use. A deeper investigation of the features with a comparison between two use types is useful to understand the characterization potential of various subsets of features. For this example, two building type classifications are compared that showed strong distinction from each other in the random forest model: university laboratories and dormitories. For this comparison, the highly comparative time-series analysis (hctsa) code repository is used as a toolkit for analysis of the generated temporal features in this study \cite{Fulcher_2013}. This toolkit has various visualization tools that enable analysis of the predictive capabilities of temporal features. Figure \ref{fig:featurecluserting_dormvslabs} shows the top forty features in differentiating university laboratories and dormitories using a simple linear classifier model. These features are clustered according to their absolute correlation coefficients in order to understand how many unique sets of informative features are present. Groups of features in the same cluster are essentially giving the same type of information about the differences between a certain set of tested classes. In the case of laboratories and dormitories, there are eight sets of clusters giving information about this distinction. The first, fourth and fifth clusters contain a couple of breakout metrics representing volatility. The second and third clusters represents magnitudes of cooling energy and consumption statistics. The sixth cluster represents seasonal metrics. The seventh cluster is a collection of fourteen features that are highly correlated, with most being related to daily ratios and consumption-related metrics. The eighth, and last cluster includes fifteen features with several also representing consumption metrics and ratios, but also several related to daily pattern frequencies.

% This toolkit includes a library of temporal features, however at this point in the analysis, only the features developed in this study are used.

%The first cluster (starting in the upper left corner of the confusion matrix) is a pair of breakout detection features measuring volatility over the long term. The next three clusters consist of a single feature; two of them describing hour-of-day metrics and one a daily pattern feature. The fifth cluster is a large group of eighteen features that are correlated with each other and all giving similar information in terms of distinction. The most prominently distinctive features amongst this set is the ratio of mean and the 95th percentile. Most of the other features in this group are also daily statistics or consumption-focused metrics. The sixth cluster encompasses features related to the residuals \emph{stl} and \emph{eemeter} models. The seventh and eighth clusters are similar to the fifth cluster in having a large number of features related to daily ratios and patterns. The conclusion to be drawn from this analysis is that these two classifications are quite distinct mostly due to consumption magnitudes and the various daily load ratios in addition to several weekly and daily pattern-based features.