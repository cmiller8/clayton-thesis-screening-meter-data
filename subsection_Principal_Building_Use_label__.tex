\subsection{Principal Building Use}
\label{sec:buildinguse}

The first scenario investigated is the characterization of primary building use type. The goal of this effort is to quantify what temporal behavior \emph{is most characteristic in a building being used for a certain purpose}. For example, what makes the electrical consumption patterns of an office building unique as compared to other uses such as a convenience store, airport, or laboratory. This objective is important in order to understand who are the \emph{peers} of a building. Whatever category a building is assigned determines what benchmark is used to determine the performance level of a building. The EnergyStar Portfolio Manager is the most common benchmarking platform in the United States and the first step in its evaluation is identifying the property type. There are 80 \emph{property types} in portfolio manager and each one is devoted to a certain primary building use type. Twenty-one of those property types are available for submission to achieve a 1-100 ENERGYSTAR score in the United States. These property types are seen in Table

\begin{table} 
    \begin{tabular}{ c }
         Bank branch  \\ 
         Barracks  \\ 
         Courthouse  \\ 
         Data center  \\ 
         Distribution center  \\ 
         Financial office  \\ 
         Hospital  \\ 
         Hotel \\ 
         K-12 school  \\ 
    \end{tabular} 
    \caption{ENERGYSTAR Score Building Types} 
\end{table}



Discriminatory features have already been visualized extensively in Section \ref{sec:featureextraction} and the differences between the primary use types are apparent in the overview heat maps of each feature. In this and the following sections, a quantification of the impact of each feature will be evaluated by using the set to predict objectives.