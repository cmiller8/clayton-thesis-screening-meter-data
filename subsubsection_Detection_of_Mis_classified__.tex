\subsubsection{Detection of Mis-classified Buildings}
\label{sec:clustering}

Previously, an example of how to characterize building use type was illustrated using a random forest model and various feature importance techniques. In this subsection, a discussion is presented of how this type of characterization can be useful in a practical setting. In the case study interviews, the topic of benchmarking of buildings was discussed. One of the issues presented to the operations teams was the concept of not having a complete understanding of . 

% Grouping of characteristically similar buildings visualize and explore the \emph{phenotypes} of buildings. The objective of this section is focused on the identification of the main modes in which in building use-type manifests itself and the identification of \emph{misfit} buildings, or those that don't seem to act like they're supposed to.