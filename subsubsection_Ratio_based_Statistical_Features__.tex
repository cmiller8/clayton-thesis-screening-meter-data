\subsubsection{Ratio-based Statistical Features}
\label{sec:ratiobasedfeatures}

After normalized consumption, the first set of temporal features to be extracted are basic statistics-based metrics that utilize the time-series data vector for various time ranges to extract information using mean, median, maximum, minimum, range, variance, and standard deviation. If a time-seies vector is described as $X$, with $N$ values of $X = {x_1, x_2,...,x_n}$, the most common statistical metric, mean (or $\mu$), can be calculated using Equation \ref{eq:mean} \cite{Mitsa_2010}.

\begin{equation}
\mu = \frac{\sum\limits_{i=1}^N x_i}{N}
\label{eq:mean}
\end{equation}

The median value for a vector is simply the middle value in an ordered set, if the number of values is odd. If the length of the vector is even, then the median is the mean of the two middle values. The minimum and maximum values are the first and last in an ordered set. Vectors of values can also be described according to percentiles. Percentiles are cutpoints dividing the range of a probability distribution based on the percentage of values below a given threshold. For example, the value at the 95\% percentile is found 95\% of the way along an ordered set, with only 5\% of the values remaining before reaching the maximum. In this section, aggregation ratios of many of these aggregation techniques are applied to the 24 hours from a single day in order to quickly characterize various types of typical behavior. The first example of these ratios is the minimum versus maximum ratio, or load ratio. This ratio is calculated by taking the daily minimum and dividing it by the daily maximum. Figure \ref{fig:loadratio_singlebuilding} illustrates a single building example of this ratio on one month of data from a case study building.