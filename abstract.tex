\chapter{Abstract}

This study focuses on the screening of characteristic data from the ever-expanding sources of raw, temporal sensor data from commercial buildings. A two-step framework is presented that extracts statistical, model-based, and pattern-based behavior from two real-world collected data sets. The collection is 507 commercial buildings extracted from various case studies and online data sources from around the world. The second are advanced metering infrastructure (AMI) data from 1,600 buildings. The goal of the framework is to reduce the expert intervention needed to utilize measured raw data in order to extract information such as building use type, performance class, and operational behavior. The first step is feature extraction and it utilizes a library of temporal data mining techniques to filter various phenomenon from the raw data. This step transforms quantitative raw data into qualitative categories that can be interpreted easily heat map visualization. In the second step, or the investigation, a supervised learning technique is tested in the ability to assign impact scores to the most important features from the first step. The efficacy of estimating variable causality of the characterized performance is tested to determine scalability amongst a heterogeneous sample of buildings. In the first set of case studies, characterization as compared to a baseline was three times more accurate in characterizing primary buildng use type, almost twice for performance class, and over four times for building operations type. For the AMI data, characterizing the standard industry class was improved by 27\% and predicting the success of energy savings measures was improved by 18\%. Qualitative insight from several campus case study interviews are discussed as well. The usefulness of the approaches was discussed in the context of campus building operations.