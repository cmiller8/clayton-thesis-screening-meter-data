\subsection{Characterization of Building Performance Class}
\label{sec:results_benchmarking}

The second objective targeted in this study is the ability for temporal features to characterize whether a building performs well or not within it use type class. Consumption is the metric being measured, therefore its not the goal of this analysis to predict the performance of a building, its to determine which temporal characteristics are correlated with good or poor performance. This effort is related to the process of benchmarking buildings. Using the insight gained through characterization of building use type, it is possible to inform whether a building's behavior matches its peers. Once a building is part of a peer group, its necessary to understand how well that building performs within that group. In this section, the case study buildings are divided according to which percentile each fits within in its in-class performance. The buildings are divided according to percentiles, with those in the lowest 33\% are classified as “Low”, the 33 to 66\% percentile are “Intermediate”, and the top 33\% are classified as “High”. As in the previous section, these classifications and a subset of temporal features are implemented into a random forest model to understand how well the features are at characterizing the different classes. Since this objective is related to consumption, all input features with known correlations to consumption were removed from the training set. These include the obvious features of consumption per area, but also include many of the statistical metrics such as maximum and minimum values. Most of the daily ratio input features remain in the analysis as they are not directly correlated with total consumption. Figure \ref{fig:performance_classification} illustrates the results of the model in an error matrix. It can be seen that \emph{high} and \emph{low} consuming buildings are well characterized. The \emph{intermediate} buildings have higher error rates and are often misclassified with the other two classes. The overall accuracy of the model for classification is 62.3\% as compared to a baseline of 38\%.

% The status quo of building performance benchmarking in the United States is the EnergyStar Rating system. This system relies on the Commerical Building Energy Consumption Survey that is completed every three years by the United States Department of Energy. In the United Kingdom, there is a mandatory building performance rating system requiring building owners to have Display Energy Certificates (DEC).



