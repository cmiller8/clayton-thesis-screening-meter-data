\subsection{Operations, Optimization, and Controls}
\label{Operations}
Unsupervised techniques focused on individual buildings themselves are placed in the category for building operations, optimization, and control. This class contains the largest number of publications, and it incorporates a wider range of applications. It is differentiated from Section \ref{AnomalyDetection} in that the applications are not as focused on detecting and fixing the anomalous behavior. This section evaluates publications within the sub-categories of occupancy detection, retrofit analysis, controls, and energy management. 

\subsubsection{Occupancy Detection}
Occupancy detection using unsupervised techniques infers human presence in a non-residential building without a labeled ground truth dataset or as part of a semi-supervised approach using a subset of labeled data. This occupancy detection is then used for analysis or as inputs for control of systems. Augello et al. \cite{augello_sensor_2011} used multiple techniques to infer occupant presence on a campus in Italy. Dong and Lam \cite{dong_building_2011} used Hidden Markov Models to detect occupancy patterns that were then used in simulation. Thanayankizil et al. \cite{thanayankizil_softgreen:_2012} developed a concept called Context Profiling in which occupancy was detected temporily and spatially. Mansur et al. \cite{mansur_learning_2015} used clustering to detect occupancy patterns from sensor data. The newest studies by Adamopoulou et al. \cite{adamopoulou_context-aware_2015} and D'Oca and Hong \cite{doca_occupancy_2015} use a range of techniques to extract rules related to occupancy. A study using wavelets illustrates the correlation of occupancy with actual energy consumption \cite{ahn_correlation_2016}.

% \subsection{Retrofit Analysis}
% The goal of approaches within Retrofit Analysis is to determine how to renovate a building using unsupervised analysis of measured data. These techniques are meant to faciliate non-intrusive retrofits that could be less expensive and labor-intensive to implements. 

\subsubsection{Controls}
Controls optimization is an enduring field of study aimed at creating a state of the best operation and energy performance for a building system such as heating, cooling, ventilation or lighting. Kusiak and Song \cite{kusiak_clustering-based_2008} created a means of optimally controlling a heating plant with clustering as a key step. Patnaik et al. \cite{patnaik_data_2010,patnaik_sustainable_2009} completed studies focused on using motif detection to find modes of chilled water plant operation that proved most optimal. Hao et al. \cite{hao_visualizing_2011} built upon these concepts to create a visual analytics tool to investigate these motifs. May-Ostendorp et al. \cite{may-ostendorp_model-predictive_2011,may-ostendorp_extraction_2013} used rule extraction as a means of enhancing a model-predictive control process of mixed-mode systems. Bogen et al. \cite{bogen_evaluating_2013} used clustering to detect usage patterns for building control system evaluation. Fan et al. \cite{fan_prediction_2013} used clustering to enhance chiller power prediction with the ultimate goal of control optimization. Hong et al. \cite{hong_towards_2013} used Empirical Mode Decomposition to spatially optimize the placement of sensors in a building. Domahidi et al. \cite{domahidi_learning_2014} used support vector machines (SVM) to extract optimized rules for supervisory control. Habib and Zucker use SAX to identify common motifs of an absorption chiller for the purpose of characterization and control \cite{habib_finding_2015}.

\subsubsection{Energy Management}
Energy management and analysis of an individual building using unsupervised techniques is becoming common due to the increasing amounts of raw building management (BMS) and energy management system (EMS) data. Users of these techniques are often facilities management professionals or consultants who undertake the process to understand how the building is consuming energy. Duarte et al. \cite{duarte_prioritizing_2011} uses visual analytics to process data from an EMS along with various pre-processing techniques. Lange et al. \cite{lange_energy_2012,lange_discovering_2013} created two overview studies focused spatio-temporal visualization of building performance data and its interpretation in various case studies. Gayeski et al. \cite{gayeski_data_2015} completed a recent survey of building operations professionals on their use of graphical interfaces of BMS and EMS dashboards. Outside of the visual analytics realm, Fan et al. \cite{fan_temporal_2015}, Xiao and Fan \cite{xiao_data_2014}, and Yu et al. \cite{yu_extracting_2013} completed studies of an entire data mining using framework using data association rules to improve operational performance.